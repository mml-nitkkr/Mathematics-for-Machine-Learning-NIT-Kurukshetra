\documentclass{article}
\newcommand{\vo}{\vec{o}\@ifnextchar{^}{\,}{}}
% \usepackage[utf8]{inputenc}
\usepackage{amsmath}
\usepackage{esvect}
\usepackage{graphics}
\usepackage{tikz}
\usepackage{libertine}
\usepackage{url}
\usepackage{amssymb}
\usepackage[
   type={CC},
        modifier={by-nc-sa},
        version={3.0}
    ]{doclicense}
    \title{Linear Algebra (Part 001)\\ Vectors and Vector Algebra}
    \author{Dr Kapil \\Department of Computer Applications\\ NIT Kurukshetra}
    
\begin{document}

    \maketitle
   Algebra of a certain domain is a study of operations available on the things (nouns) of the domain.\\
Example. In cooking, things are-
\begin{itemize}
    \item Edible material (for example spices, flour, water, milk etc.)
    \item Utensils
    \item Burner etc.
\end{itemize}

And the operations are-
\begin{itemize}
    \item Pouring
    \item Mixing
    \item Putting something on an ignited burner etc.
    \item Keeping
    \item Throwing
\end{itemize}
Now, kitchen algebra may mean studying various sequences of the operations and things (in mathematics we say them symbols).\\
Now you can notice that
\begin{itemize}
    \item Some operations are defined only with certain symbols. Like you do not put Glass on an ignited burner.
    \item Sometimes you mix only two things (binary) at a time while other times you put a lot of things together (n-ary) and then mix. Some operations may be unary only that is they involve only one thing for example throwing.
    \item Sometimes when you do some operation with edible material and it becomes inedible.
    Other times you do some operation with inedible material and it becomes edible.
\end{itemize}

Formally, the definition is-  Algebra is the study of mathematical symbols and the rules for manipulating these symbols; it is a unifying thread of almost all of mathematics. It includes everything from elementary equation solving to the study of abstractions such as groups, rings, and fields (\href{https://en.wikipedia.org/wiki/Algebra}{Wikipedia}).

In Linear Algebra, we study algebra of vectors. Here symbols or things are vectors represented as ($\overrightarrow{x}$ or \textbf{$x$}). Geometric Vectors, that we have seen already
Example: Velocity , force  and  acceleration  etc.
As applied  to vectors  we  have  scalar  quantities that has only magnitudes For example temperature,  energy etc.\\
We have worked with two dimensional (2D) and three dimensional (3-D) vectors during our twelfth standard. You might probably remember that $2\hat{i}+3\hat{j}, (4,5),
        \begin{pmatrix}
         3 \\
         5 \\
       \end{pmatrix}
        $ are different ways to represent the vectors of 2-D.\\
        And that the 3-D vectors are represented as
        $(2,1,5), 2\hat{i}+5\hat{j}+3\hat{k},
          \begin{pmatrix}
          2\\
          7\\
          3
          \end{pmatrix}
         $\\
We also perform many operations on them such as
\begin{enumerate}
    \item Addition on two vectors is defined as $ \begin{pmatrix}
  a\\
  b\\
  c
 \end{pmatrix}
 +
 \begin{pmatrix}
 d\\
 e\\
 f
\end{pmatrix}
=
 \begin{pmatrix}
a+d\\
b+e\\
c+f
\end{pmatrix}$. For example\\
            \begin{enumerate}
                 \item
                    $\begin{pmatrix}
                      2\\                   3\\
                     \end{pmatrix}
                      +
                     \begin{pmatrix}
                        5\\
                        -2\\
                    \end{pmatrix}
                    =
                    \begin{pmatrix}
                        7\\
                        1\\
                    \end{pmatrix}$\\
                \item
                    $\begin{pmatrix}
                    1\\
                    -7\\
                    2
                    \end{pmatrix}
                    +
                    \begin{pmatrix}
                        3\\
                        4\\
                        3
                    \end{pmatrix}
                     =
                    \begin{pmatrix}
                    4\\
                    -3\\
                    5
                    \end{pmatrix}$\\
            \end{enumerate}



\item Scalar Multiplication scales the length of the vector but the  direction is retained. It is an operation between some scalar value and a vector. Mathematically, we define as if $a\in \mathbb{R}$ then
$a*\begin{pmatrix}
b\\c\\d

\end{pmatrix}=
\begin{pmatrix}
a*b\\a*c\\a*d
\end{pmatrix}$. For example-
\begin{enumerate}
    \item $2*\begin{pmatrix}
4\\
3
\end{pmatrix}=
\begin{pmatrix}
8\\
6
\end{pmatrix}$\\

\item $-1/2*\begin{pmatrix}
7\\3\\10
\end{pmatrix}=
\begin{pmatrix}
7*\frac{-1}{2}\\3*\frac{-1}{2}\\10*\frac{-1}{2}
\end{pmatrix}=
\begin{pmatrix}
\frac{-7}{2}\\\frac{-3}{2}\\\frac{-10}{2}
\end{pmatrix}$\\

\end{enumerate}
\end{enumerate}

Both of the stated operations have geometrical interpretations can be seen from class XI and XII books.\\
Exercise: Do following operations on graph (graphically) and verify with above stated theoretically
\begin{itemize}
    \item Addition of two Vectors with parallelogram law on the graph and share. Draw $\begin{pmatrix}
        2\\3
    \end{pmatrix},
    \begin{pmatrix}
        1\\-1
    \end{pmatrix}
     and \begin{pmatrix}
        2+1\\3-1
    \end{pmatrix}$
    \item Multiply a given vector with different scalars and then draw the two vectors. For example draw $\begin{pmatrix}
        2\\3
    \end{pmatrix},2*\begin{pmatrix}
        2\\3
    \end{pmatrix},
    -1*\begin{pmatrix}
        2\\3
    \end{pmatrix},
    \frac{1}{2}*\begin{pmatrix}
        2\\3
    \end{pmatrix}$ and check if there is some pattern in the same. You can try with more examples by restricting the value of scalar, c like $c\in [0,1]$ or $[-1,0]$ or $[1,\infty)$ or $(-\infty,-1]$ etc.
\end{itemize}

There are few more operations which we are covering here just for sake of introduction as they are not basic operation. But they will appear again in detail.\\
\begin{itemize}
    \item Subtraction is defined as $\begin{pmatrix}
  a\\
  b\\
  c
 \end{pmatrix}
 -
 \begin{pmatrix}
 d\\
 e\\
 f
\end{pmatrix}
=
 \begin{pmatrix}
a-d\\
b-e\\
c-f
\end{pmatrix}$. It is not a basic operation as it can expressed as $vector_1 + (-1)* vector_2$. A few examples are-
\begin{enumerate}
    \item $\begin{pmatrix}
1\\
2
\end{pmatrix}-
 \begin{pmatrix}
3\\
4
\end{pmatrix}
=
 \begin{pmatrix}
-2\\
-2
\end{pmatrix}$\\
\item  $\begin{pmatrix}
1\\
-7\\
3
\end{pmatrix}-
\begin{pmatrix}
2\\
8\\
6
\end{pmatrix}=
\begin{pmatrix}
-1\\
-15\\
-3
\end{pmatrix}$\\
\end{enumerate}


 \item Magnitude of a vector, $\overrightarrow{v}\in\mathbb{R}^n$ is denoted as
$\|\overrightarrow{v}\|$ and  is  defined as $\|{\overrightarrow{v}\|} =  \sqrt{\sum_{i=1}^{n} {v_i}}$. For example
\begin{itemize}
    \item $\overrightarrow{v}=\begin{pmatrix}1\\4\\-3\end{pmatrix}$, then 
$\|\overrightarrow{v}\|=\sqrt{(1^2+4^2+(-3)^2)} = \sqrt{(26)}$\\

    \item for 2D vector, $\overrightarrow{u}$ = \begin{bmatrix} 2\\7\\ \end{bmatrix}\\
$\|{\overrightarrow{u}\|^2} = 2^2+7^2=4+49=53=53$\\
$\|{\overrightarrow{u}\|} = \sqrt{53}$\\
\end{itemize}
Magnitude of  any vector is " non- negative" value.  $\\
$Hopefully you know what does non negative means. $\\
$ If a is non negative , it means   $a>0$ or $a=0$  \\
$ While a is positive means$, $ a$>0 $i.e. a cannot  be zero $\\
\item Dot  Product: Dot product of two  vectors $ \overrightarrow{v},\overrightarrow{u} $is defined as $\\ Let
\overrightarrow{u}= \begin{pmatrix}1\\2 \end{pmatrix}  
\overrightarrow{v}=\begin{pmatrix} 4\\-2 \end{pmatrix}\\
\overrightarrow{u}*\overrightarrow{v}= 1*4+2*(-2)=4-4=0\\
$i.e.in other words if \overrightarrow{u}=\begin{pmatrix} u1\\u2\\u3 \end{pmatrix}
and  \overrightarrow{v}= \begin{pmatrix} v1\\v2\\v3 \end{pmatrix}\\

Then \overrightarrow{u}*\overrightarrow{v}=u1*v1+u2*v2+u3*v3 \\
 $ Above operation and vectors are basic of  this course .\\
 As we have  seen geometric vectors  , we can also interpret some\\ other objects, specially mathematical objects as vectors. $\\
 (a) $a line  is represented as \\
 y=mx+c can be thought of as a point in 2D space with (m,c)
as coordinates\\
(b)$ a polynomial can also be treated as vector$ \\
 P(x)=2*x^2-3*x+5, $can be  thought of  as a vector$ \\
\begin{pmatrix}
2\\-3\\5\\
\end{pmatrix}\\
(c) $An image is made up of  pixels with intensity values of colour channels Red , Green and Blue$ \\
\begin{left}
\begin{tikzpicture}
\draw[ultra thick, red] (0,0) rectangle (2,2)
\draw[ultra thick, green] (.5,.5) rectangle (2.5,2.5)
\draw[ultra thick, blue] (1,1) rectangle (3,3)
\end{tikzpicture}
\end{left}
$ So  for  any image of size 1024*1024 resolution then it\\
has ~ 1024 k pixels and thus each channels informatics is stored in 1 MB \\
 And for 3 channels 3 MB space is required .\\
 So, each image is a vector or collection of 3 million numbers $\\
 $What is most important thing about  these vectors . All these vectors have  the speciality that after manipulations, the  resultant things  are  also vectors and are of some  style as that  of  input vectors.\\$
i.e. (i)\begin{pmatrix}
1\\7\\2
\end{pmatrix}\in R^3  $ 3D Vectors\\
then c \overrightarrow{u}=2*\begin{pmatrix}
1\\7\\2
\end{pmatrix}= \begin{pmatrix}
2\\14\\4
\end{pmatrix}\\
(ii) \overrightarrow{u}= \begin{pmatrix}
1\\7\\2
\end{pmatrix},
  \overrightarrow{u}= \begin{pmatrix}
-2\\3\\-4
\end{pmatrix}\in R^3\\
\overrightarrow{u}+\overrightarrow{v}= \begin{pmatrix}
-1\\10\\-2
\end{pmatrix}\in R^3 \\
$ In such cases where input symbols(before being manipulation) &\\ output symbols(after being operated)belong to same collection or\\ set. \\
You can  notice that adding any two polynomials will result into\\ another polynomial only. \\
 Also , if we multi;ly a polynomial with some constant value then\\ also the  resultant is polynomial.$\\
P1(x)= 2x^2+3x+1\\
P2(x)=5x-7 \\
P1(x)*P2(x)=2x^2+8x-6 $ is also a polynomial $\\
2*P1(x)=4x^2+6x+2 $ is also a polynomial $\\
(*)$ So let we define  Vectors as - \\
 $\rightarrow$ A set of things when added among themselves , the\\ product remain in the same set.\\
 And when the such thing is multiplied by certain real or complex \\value , then also the product remains in the same set.$\\
  Set= Collection\\
$ (*)Elements  of R^n $(R: the set of  real numbers).In other words$\\
 $ tuples of  n real numbers are vectors\\
  It follows these two properties.$\\
 (i)$\forall\overrightarrow{u},\overrightarrow{v}\ \in R^n \\
 (ii)$\forall c \in R $ and $  $\forall \overrightarrow{u}\in R^n \\
  c \overrightarrow{u}\in R^n \\
$So ,  here we will try to understand behaviour of these \\ vectors(R^n).$ But you will see all vectors cab be represented using\\ R^n type vectors $which essentially tells that all the results of\\ R^n$ are also applicable on all the vectors we are  talking about.$

\end{document}