to\documentclass{article}

\usepackage[margin = 3cm, footskip = 30pt]{geometry}
\usepackage{amsmath}
\usepackage{tikz}
\usepackage{amssymb}
\usepackage{pgfplots}






\title{Linear Algebra (Part 002)\\Understanding Simultaneous System of Linear Equations}
\author{Dr Kapil\\Department of Computer Applications\\ NIT Kurukshetra}
\date{\today}

\begin{document}

\maketitle


In the first part, we tried to understand vector.\\

The property of retaining the same set after being operated is called ``closure`` property.~
In other words we say $R^n$ is closed over addition and scalar multiplication.\\

Now, let us see these vector in some real problems.\\

\section{System of Linear Equations}\label{first_section}
    Example : \\

A company produces $N_1$, ..., $N_n$ for which resource $R_1$, ..., $R_m$ are required. To produce one unit of product $N_j$, $a_i{}_j$ unit of resources $R_i$ are required, where $1 \leq i \leq m$ \& $1 \leq j \leq n$. It is known that we have exactly $b_i$ units of resources $R_i$. So what could be the plan i.e. quantity of products should be produced so that no wastage is left over i.e. nothing is left behind.\\

Let $x_j$ be the number of units produced according to the plan for the product $N_j$. Thus, the total consumption of $R_i$ in producing $a_{ij}x_j$. And total consumption of $R_i$ in producing $x_1,x_2,...,x_n$ is $a_i_1x_1 + a_i_2x_2 + ... + a_i_nx_n$. To ensure no wastage, we have to equate it with total available quantity of resource $R_i$ (i.e. $b_i$). Hence,

\begin{align}
   a_i_1x_1 + a_i_2x_2 + ... + a_i_nx_n & = b_i
\end{align}

Sum of usage of all the units of resources \(R_i\) in the production of \( N_1, N_2, ..., N_m\)\\

i.e.\\
\begin{align}
    a_1_1x_1 + a_1_2x_2 + ... + a_1_nx_n &= b_1\nonumber\\
    a_2_1x_1 + a_2_2x_2 + ... + a_2_nx_n &= b_2\nonumber\\
    .\nonumber\\
    .\nonumber\\
    .\nonumber\\
    a_m_1x_1 + a_m_2x_2 + ... + a_m_nx_n &= b_n\nonumber\\
\end{align}

When a particular value of \((x_1, x_2, ..., x_n)\) solves all the equations that solution of system. This is also known as simultaneous set of linear equations. That is all the equality's
should hold for same value of \((x_1, x_2, ..., x_n) \in R^n\).\\

Now, instead of trying to understand such a big system. Let us talk about smaller \& concrete system.\\
Case 1:\\
\begin{align}
    x_1 + x_2 &= 2 &&  \text{(L1)}\\
    x_1 - x_2 &= 0 &&  \text{(L2)}
\end{align}

Clearly both the equality's are true when \(x_1\) = 1 \& \(x_2\) = 1.\\

If you see them graphically any point on \(L_1\) satisfies equation(3) \& any point on \(L_2\) satisfies equation(4). And the intersection point is the one which satisfies both equation(3) and equation(4).\\

\begin{tikzpicture}
    \begin{axis}[
        axis lines = left,
        xlabel = $x_1$,
        ylabel = {$x_2$},
        xmin=0, xmax=7,
        ymin=0, ymax=7,
    ]
        \addplot [color=red]{2-x};
        \addlegendentry{$x_1 + x_2 = 2$}
        \addplot [color=blue]{x};
        \addlegendentry{$x_1 - x_2 = 0$}
    
    \end{axis}
\end{tikzpicture}

Case 2:\\

\begin{align}
    x_1 + x_2 &= 2 &&  \text{(L1)}\\
    2x_1 &= 4 - 2x_2 &&  \text{(L2)}
\end{align}

Keeping $x_1, x_2$ on the same side of equality \& dividing the equation(4) by 2 throughout, we get to know that these are same equation's. Therefore, any point (a, 2-a) \forall a \in \(R^n\) will satisfy both the equations. And this `a` can take infinitely many value. So, we have infinitely many solutions. And graphically they mean both the line overlap \& hence provide infinite number of solution that can satisfy both equation simultaneously.\\

\begin{tikzpicture}
    \begin{axis}[
        axis lines = left,
        xlabel = $x_1$,
        ylabel = {$x_2$},
        xmin=0, xmax=7,
        ymin=0, ymax=7,
    ]
        \addplot [color=red]{2-x};
        \addlegendentry{$L_1, L_2$}
    \end{axis}
\end{tikzpicture}

Case 3:\\

\begin{align}
    x_1 + x_2 &= 2 &&  \text{(L1)}\\
    2x_1 + 2x_2 &= 5  &&  \text{(L2)}
\end{align}

Obviously, when you multiply first by 2 then for any value of \((x_1, x_2)\), satisfy equation(7) cannot satisfy equation(8).\\

Thus, we cannot have a common solution among them. Graphically, it means you can see, they can not meet.\\

\begin{tikzpicture}
    \begin{axis}[
        axis lines = left,
        xlabel = $x_1$,
        ylabel = {$x_2$},
        xmin=0, xmax=7,
        ymin=0, ymax=7,
    ]
        \addplot [color=red]{2-x};
        \addlegendentry{$L_1$}
        \addplot [color=blue]{(5 - 2*x) / 2};
        \addlegendentry{$L_2$}
    \end{axis}
\end{tikzpicture}

You can try to generate the similar case for 3 variable equations. [refer to book page 20.]\\

Conclusion:\\

Any real valued system of linear equations has either no, exactly one, or infinitely many solutions. Now, let us look case 1 int different way.\\
\begin{align}
    x_1 + x_2 &= 2 \nonumber\\
    x_1 - x_2 &= 0 \nonumber
\end{align}

\[
\implies x_1 \begin{pmatrix}
                1\\
                1\\
             \end{pmatrix} + x_2 \begin{pmatrix}
                                    1\\
                                    -1\\
                                 \end{pmatrix} &= \begin{pmatrix}
                                                    2\\
                                                    0\\
                                                  \end{pmatrix}
\]\\

We have only one equation but involving 2D vector \(\vec{v_1}\) \& \(\vec{v_2}\). Now, what does this mean?\\
Let's go back to our vector operation\\
\begin{enumerate}
    \item scalar multiplication - It scales the vector by some scalar quantity\\
\[  ex.~ \vec{u_1} = \begin{pmatrix}
                                1\\
                                2\\
                            \end{pmatrix}, ~ 2* \vec{u} \begin{pmatrix}
                                                            1\\
                                                            2\\
                                                      \end{pmatrix} &= \begin{pmatrix}
                                                                            2\\
                                                                            4\\
                                                                       \end{pmatrix}
\]

\begin{tikzpicture}
    \begin{axis}[
        axis lines = left,
        xlabel = $x_1$,
        ylabel = {$x_2$},
        xmin=0, xmax=7,
        ymin=0, ymax=7,
    ]
        \addplot coordinates {(0,0) (1, 2) (2, 4)};
    \end{axis}
\end{tikzpicture}


Graphically, i.e. it is just changing the magnitude of vector.
    
    \item Addition of two vector 
    
\[
    \vec{u} &= \begin{pmatrix}
                    1\\
                    1\\
               \end{pmatrix}, \vec{v} &= 
               \begin{pmatrix}
                    1\\
                    -1\\
               \end{pmatrix}
\]    




\begin{tikzpicture}
    \begin{axis}[
        axis lines = left,
        xlabel = $x_1$,
        ylabel = {$x_2$},
        xmin=-5, xmax=5,
        ymin=-5, ymax=5,
    ]
        \addplot coordinates {(0,0) (1, 1) (2, 0)};
        \addplot coordinates {(0,0) (1,-1)};
    \end{axis}
\end{tikzpicture}
\end{enumerate}

Put the tail of one vector on the head of another and now whatever location you are pointing to, is the resultant vector.\\

Now going back to case 1\\

\[
    x_1 \begin{pmatrix}
            1\\
            1\\
        \end{pmatrix} + x_2 \begin{pmatrix}
                                1\\
                                -1\\
                            \end{pmatrix} &= \begin{pmatrix}
                                                2\\
                                                0\\
                                             \end{pmatrix}
\]

So, this equation asks for what scaling factor of the vectors on addition can generate a new \\ vector \( \begin{pmatrix}
                2\\
                0\\
              \end{pmatrix} \) .\\

The sequence of scalar multiplication \& addition is called Linear Combination.\\

So, the question what is the right Linear Combination of \begin{pmatrix}
                                                            1\\
                                                            1\\
                                                        \end{pmatrix} \& \begin{pmatrix}
                                                                            1\\
                                                                            -1\\
                                                                        \end{pmatrix} that can make \begin{pmatrix}
                                                        2\\
                                                        0\\
                                                     \end{pmatrix}.
                                                     
Further you can see in following way \begin{pmatrix}
                                        1 & 1\\ 
                                        1 & -1\\ 
                                     \end{pmatrix} \begin{pmatrix}
                                                        x_1\\
                                                        x_2\\
                                                   \end{pmatrix} &= \begin{pmatrix}
                                                                        2\\
                                                                        0\\
                                                                     \end{pmatrix}


\(A\vec{x} = \vec{b}\),~ A = \begin{pmatrix}
                            1 && 1\\
                            1 && -1\\
                        \end{pmatrix}, a matrix of order 2*2 \vec{x} is a unknown vector 2*1 \vec{b} know vector.




\section{Exercise (Homework)}

Extend Linear Combination view (or Understanding) for the other two cases.

\end{document}