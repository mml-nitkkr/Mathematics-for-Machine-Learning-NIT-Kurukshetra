\documentclass[11pt,a4paper]{article}

\usepackage{lipsum} 
\usepackage{graphicx}

\usepackage{geometry}
 \geometry{
 a4paper,
 total={170mm,257mm},
 left=20mm,
 top=20mm,
 }
 
\usepackage{fancyhdr}
\cfoot{
\hspace{0.5cm}
\includegraphics[width=0.2\textwidth]{CC-BY-NC-SA.pdf}}

\renewcommand{\headrulewidth}{0pt}
\renewcommand{\footrulewidth}{0pt}
\setlength\headheight{80.0pt}
\addtolength{\textheight}{-80.0pt}

\usepackage{tikz}
\usetikzlibrary{shapes.geometric, arrows,positioning,automata}


\usepackage{amsmath,amssymb,amsfonts}
\usepackage{blkarray}
\usepackage{enumerate}

\begin{document}
\title{Linear Algebra (Part 004c)\\Solved Example of system of linear equations}
\author{Dr Kapil\\kapil $@$ nitkkr $\cdot$ ac $\cdot$ in\\Department of Computer Applications\\ NIT Kurukshetra}
\date{\today}
\maketitle
\thispagestyle{fancy}   

\textbf{Question}\\ 
Solve the following simultaneous system of Linear Equation using Gauss Jordon method. Also state all the elimination matrices (and other matrices)\\

\textbf{Answer}
\[
\begin{split}
    2x_1 - 3x_2 + 4x_3 - x_4 + 0x_5 &= 9\\
    4x_1 - 6x_2 - 8x_3 - x_4 + 0x_5 &= 17\\
    3x_1 + 2x_2 - 3x_3 + x_4 + x_5 &= -3\\
    5x_1 - x_2 + x_3 + 0x_4 + x_5 &= 6
\end{split}
\]
\[
\text{A} = 
    \begin{blockarray}{cccccc}
        x_1 & x_2 & x_3 & x_4 & x_5 & b&\\
        \begin{block}{[ccccc|c]}
            2 & -3 & 4 & -1 & 0 & 9\\
            4 & -6 & 8 & -1 & 0 & 17\\
            3 & 2 & -3 & 1 & 1 & -3\\
            5 & -1 & 1 & 0 & 1 & 1
        \end{block}
    \end{blockarray}\] 
    \[\begin{matrix}
        \xrightarrow{ \tikz[remember picture]\node[inner sep=4pt] (A) {};R_2 \leftarrow R_2 - 2R_1}\\
        \xrightarrow{\tikz[remember picture]\node[inner sep=4pt] (B) {}; R_3 \leftarrow R_3 - \frac{3}{2}R_1}\\
        \xrightarrow{\tikz[remember picture]\node[inner sep=4pt] (C) {}; R_4 \leftarrow R_4 - \frac{5}{2}R_1}
    \end{matrix}
    \left[{\begin{array}{ccccc|c}
        2 & -3 & 4 & -1 & 0 & 9\\
        0 & 0 & 0 & 1 & 0 & -1 \\
        0 & 6.5 & -9 & 2.5 & 1 & -16.5\\
        0 & 6.5 & -9 & 2.5 & 1 & -16.5
   \end{array}}\right]\\
   
\vspace{2cm}
\tikz[remember picture]\node[inner sep=0pt] (a) {};E_{21} = \begin{bmatrix}
            1 & 0 & 0 & 0\\
            -2 & 1 & 0 & 0\\
            0 & 0 & 1 & 0\\
            0 & 0 & 0 & 1
         \end{bmatrix}, ~~~~\tikz[remember picture]\node[inner sep=0pt] (b) {};E_{31} = \begin{bmatrix}
                                    1 & 0 & 0 & 0\\
                                    0 & 1 & 0 & 0\\
                                    \frac{-3}{2} & 0 & 1 & 0\\
                                    0 & 0 & 0 & 1
                                 \end{bmatrix}, ~~~~~\tikz[remember picture]\node[inner sep=0pt] (c) {};E_{42} = \begin{bmatrix}
                                                            1 & 0 & 0 & 0\\
                                                            0 & 1 & 0 & 0\\
                                                            0 & 0 & 1 & 0\\
                                                            \frac{-5}{2} & 0 & 0 &1
                                                       \end{bmatrix}
\tikz[overlay, remember picture]\draw[->,>=stealth] (A.west) -- +(-4.5,0) |-(a.north);
\tikz[overlay, remember picture]\draw[->,>=stealth] (B) -- +(-0.4,0) |- (b.north);
\tikz[overlay, remember picture]\draw[->,>=stealth] (C.south) |- +(0,-2)-- +(4.7,-2) |- (c.north);
\]\newpage
\[
E_{41}E_{31}E_{21}A \xrightarrow{R_4 \leftarrow\tikz[remember picture]\node[minimum size= 5pt,inner sep=0pt] (D) {};R_4 - R_3}\left[{\begin{array}{ccccc|c}
                                    2 & -3 & 4 & -1 & 0 & 9\\
                                    0 & 0 & 0 & 1 & 0 & -1 \\
                                    0 & 6.5 & -9 & 2.5 & 1 & -16.5\\
                                    0 & 0 & 0 & 0 & 0 & 0
                                    \end{array}}\right] 
                                                                    
\tikz[remember picture]\node[inner sep=2pt](d) {};E_{42}=\begin{bmatrix}
            1 & 0 & 0 & 0\\ 
            0 & 1 & 0 & 0\\ 
            0 & 0 & 1 & 0\\ 
            0 & 0 & -1 & 1 
         \end{bmatrix} \hspace{5cm} E_{42}E_{41}\tikz[remember picture]\node[inner sep=0pt] (E) {};E_{31}E_{21}A
         
\tikz[overlay, remember picture]\draw[->,thick,>=stealth] (D.north) -- +(0,1) -| (d.north);
\tikz[overlay, remember picture]\draw[->,thick,>=stealth] (E.north) -- +(0,1);
         \]

Rearranging the columns is necessary to get 1 pivot
\[
    \begin{bmatrix}
        2 & -1    & 4  & -3  & 0\\
        0 & 1     & 0  & 1   & 0\\
        0 & 2.5   & -9 & 6.5 & 1\\
        0 & 0     & 0  & 0   & 0 
    \end{bmatrix} \begin{bmatrix}
                    x_1 \\ x_4 \\ x_3 \\ x_2 \\ x_5
                 \end{bmatrix} = \begin{bmatrix}
                                    9 \\ -1 \\ -16.5 \\ 0
                                 \end{bmatrix} \] Which is,
\[                                 
\left[{\begin{array}{ccccc|c}
        2 & -1    & 4  & -3  & 0 & 9\\
        0 & 1     & 0  & 1   & 0 & -1\\
        0 & 2.5   & -9 & 6.5 & 1 & -16.5\\
        0 & 0     & 0  & 0   & 0 & 0
\end{array}}\right]
\]
The above operation is column operation,
So the matrix to be multiplied is on the right side of it i.e.\\
\[
(\begin{bmatrix}
   2 & -3 & 4 & -1 & 0\\
   0 & 0 & 0 & 1 & 0\\
   0 & 6.5 & -9 & 2.5 & 1\\
   0 & 0 & 0 & 0 & 0
\end{bmatrix} \begin{bmatrix}
                1 & 0 & 0 & 0 & 0\\
                0 & 0 & 0 & 1 & 0\\
                0 & 0 & 1 & 0 & 0\\
                0 & 1 & 0 & 0 & 0\\
                0 & 0 & 0 & 0 & 1
              \end{bmatrix})( \begin{bmatrix}
                                1 & 0 & 0 & 0 & 0\\
                                0 & 0 & 0 & 1 & 0\\
                                0 & 0 & 1 & 0 & 0\\
                                0 & 1 & 0 & 0 & 0\\
                                0 & 0 & 0 & 0 & 1
                              \end{bmatrix} \begin{bmatrix}
                                            x_1 \\ x_2 \\ x_3 \\ x_4 \\x_5
                                             \end{bmatrix} )
= \begin{bmatrix}
9\\-1\\-16.5\\0
\end{bmatrix}\\

E_12\\

R_1 \longleftarrow R_1 + R_2\\

R_3 \longleftarrow R_3 - 2.5 R_2
~~~~~\left[{\begin{array}{ccccc|c}
2 & 0 & 4 & -3 & 0 & 8\\
0 & 1 & 0 & 0 & 0 & -1\\
0 & 0 & -9 & 6.5 & 1 & -1 \\
0 & 0 & 0 & 0 & 0 & 0
\end{array}}\right] , \\

E_{12} = \begin{bmatrix}
        1 & 1 & 0 & 0\\
        0 & 1 & 0 & 0\\
        0 & 0 & 1 & 0\\
        0 & 0 & 0 & 1 \end{bmatrix}, E_{32} = \begin{bmatrix}
                                                1 & 0 & 0 & 0\\
                                                0 & 1 & 0 & 0\\
                                                0 & -2.5 & 1 & 0\\
                                                0 & 0 & 0 & 1
                                                \end{bmatrix}
\]
\newpage

For simplification we can further exchange its column $3^rd$ with $5^th$\\
\[
\begin{bmatrix}
2 & 0 & 0 & -3 & 4\\
0 & 1 & 0 & 0 & 0\\
0 & 0 & 1 & 6.5 & -9\\
0 & 0 & 0 & 0 & 0
\end{bmatrix}
\begin{bmatrix}
x_1\\x_4\\x_5\\x_2\\x_3
\end{bmatrix} = 
\begin{bmatrix}
8\\-1\\-14\\0
\end{bmatrix}

R_1 \Longleftarrow \frac{R_1}{2}\\

\begin{bmatrix}
\frac{1}{2} & 0 & 0 & 0\\
0 & 1 0 & 0\\
0 & 0 & 1 & 0\\
0 & 0 & 0 & 1
\end{bmatrix}
\longrightarrow
\left[{\begin{array}{ccc|cc|c}
1 & 0 & 0 & \frac{-3}{2} & 2 & 4\\
0 & 1 & 0 & 0 & 0 & -1\\
0 & 0 & 1 & 6.5 & -9 & -14 \\
\hline
0 & 0 & 0 & 0 & 0 & 0
\end{array}}\right]\\

\left[{\begin{array}{c|c|c}
I & F & b\\ \hline
0 & 0 & 0\\
\end{array}}\right]

\text{Basic variables}~~~ \Vec{x}~~~~ \begin{bmatrix}
                                        x_1\\x_4\\x_5
                                    \end{bmatrix}       
\text{Free variables}~~~ \Vec{y}~~~~ \begin{bmatrix}
                                        x_2\\x_3
                                    \end{bmatrix}  \\
                                    
\therefore \begin{bmatrix}
            \Vec{x}\\\Vec{y}
            \end{bmatrix} =
\begin{bmatrix}
4\\-1\\-14\\0\\0
\end{bmatrix} + 
\begin{bmatrix}
\frac{3}{2} & -2\\
0 & 0\\
-6.5 & 9\\
1 & 0 \\
0 & 1
\end{bmatrix} 
\begin{bmatrix}
x_2\\x_3
\end{bmatrix}\\

\text{or}~~~~
\begin{bmatrix}
x_1\\x_2\\x_3\\x_4\\x_5
\end{bmatrix} = 
\begin{bmatrix}
4 \\ 0 \\ 0 \\ -1 \\ -14
\end{bmatrix} + 
\begin{bmatrix}
\frac{3}{2} & -2\\
1 & 0\\
0 & 1\\
0 & 0\\
-6.5 & 9
\end{bmatrix}

\]
\end{document}
