\documentclass[11pt,a4paper]{article}

\usepackage{lipsum}
\usepackage{graphicx}

\usepackage{fancyhdr}
%\pagestyle{fancy}
\cfoot{
\vspace{1mm}\hspace{0.5cm}
\includegraphics[width=0.2\textwidth]{CC-BY-NC-SA.pdf}}

\renewcommand{\headrulewidth}{0pt}
\renewcommand{\footrulewidth}{0pt}
\setlength\headheight{80.0pt}
\addtolength{\textheight}{-80.0pt}

\usepackage{tikz}
\usetikzlibrary{shapes.geometric, arrows,positioning,automata}


\usepackage{amsmath,amssymb,amsfonts}
\usepackage{blkarray}
\usepackage[table]{xcolor}
\usepackage{enumerate}

\begin{document}
\title{Linear Algebra (Part 006)\\Solved Example of system of linear equations}
\author{Dr Kapil\\kapil $@$ nitkkr $\cdot$ ac $\cdot$ in\\Department of Computer Applications\\ NIT Kurukshetra}
\date{\today}
\maketitle
\thispagestyle{fancy}   

$\boldsymbol{Question}$. 

Solve the following simultaneous system of Linear Equation using Gauss Jordon method. Also state all the elimination matrices (\& other matrices)

\[
\begin{split}
    2x_1 - 3x_2 + 4x_3 - x_4 + 0x_5 &= 9\\
    4x_1 - 6x_2 - 8x_3 - x_4 + 0x_5 &= 17\\
    3x_1 + 2x_2 - 3x_3 + x_4 + x_5 &= -3\\
    5x_1 - x_2 + x_3 + 0x_4 + x_5 &= 6
\end{split}
\]
\[
\text{A} = 
    \begin{blockarray}{cccccc}
        x_1 & x_2 & x_3 & x_4 & x_5 & &\\
        \begin{block}{[ccccc|c]}
            2 & -3 & 4 & -1 & 0 & 9\\
            4 & -6 & 8 & -1 & 0 & 17\\
            3 & 2 & -3 & 1 & 1 & -3\\
            5 & -1 & 1 & 0 & 1 & 1
        \end{block}
    \end{blockarray}
    \begin{matrix}
        \xrightarrow{ R_2 \leftarrow R_2 - 2R_1}\\
        \xrightarrow{ R_3 \leftarrow R_3 - \frac{3}{2}R_1}\\
        \xrightarrow{ R_4 \leftarrow R_4 - \frac{5}{2}R_1}
    \end{matrix}
    \begin{bmatrix}[ccccc|c]
        2 & -3 & 4 & -1 & 0 & 9\\
        0 & 0 & 0 & 1 & 0 & -1 \\
        0 & 6.5 & -9 & 2.5 & 1 & -16.5\\
        0 & 6.5 & -9 & 2.5 & 1 & -16.5
   \end{bmatrix}
\]

\[
E_{21} = \begin{bmatrix}
            1 & 0 & 0 & 0\\
            -2 & 1 & 0 & 0\\
            0 & 0 & 1 & 0\\
            0 & 0 & 0 & 1
         \end{bmatrix}, ~~~~E_{31} = \begin{bmatrix}
                                    1 & 0 & 0 & 0\\
                                    0 & 1 & 0 & 0\\
                                    \frac{-3}{2} & 0 & 1 & 0\\
                                    0 & 0 & 0 & 1
                                 \end{bmatrix}, ~~~~~E_{42} = \begin{bmatrix}
                                                            1 & 0 & 0 & 0\\
                                                            0 & 1 & 0 & 0\\
                                                            0 & 0 & 1 & 0\\
                                                            \frac{-5}{2} & 0 & 0 & 1
                                                       \end{bmatrix}
\]
\[
E_{41}E_{31}E_{21}A \xrightarrow{R_4\leftarrow R_4 - R_3}\begin{bmatrix}[ccccc|c]
                                                                        2 & -3 & 4 & -1 & 0 & 9\\
                                                                        0 & 0 & 0 & 1 & 0 & -1 \\
                                                                        0 & 6.5 & -9 & 2.5 & 1 & -16.5\\
                                                                        0 & 0 & 0 & 0 & 0 & 0
                                                                   \end{bmatrix} 
\]

\[
E_{42} = \begin{bmatrix}
            1 & 0 & 0 & 0\\ 
            0 & 1 & 0 & 0\\ 
            0 & 0 & 1 & 0\\ 
            0 & 0 & -1 & 1 
         \end{bmatrix}
\]
Rearranging the columns is necessary to get 1 pivot
\[
    \begin{bmatrix}
        2 & -1    & 4  & -3  & 0\\
        0 & 1     & 0  & 1   & 0\\
        0 & 2.5   & -9 & 6.5 & 1\\
        0 & 0     & 0  & 0   & 0 
    \end{bmatrix} \begin{bmatrix}
                    x_1 \\ x_4 \\ x_3 \\ x_2 \\ x_5
                 \end{bmatrix} = \begin{bmatrix}
                                    9 \\ -1 \\ -16.5 \\ 0
                                 \end{bmatrix}
\]
from back 
\[
\begin{bmatrix}[ccccc|c]
        2 & -1    & 4  & -3  & 0 & 9\\
        0 & 1     & 0  & 1   & 0 & -1\\
        0 & 2.5   & -9 & 6.5 & 1 & -16.5\\
        0 & 0     & 0  & 0   & 0 & 0
\end{bmatrix}
\]
The above operation is column operation, So the matrix to be multiplied is on the right side of it i.e.
\[
(
\begin{bmatrix}
   2 & -3 & 4 & -1 & 0\\
   0 & 0 & 0 & 1 & 0\\
   0 & 6.5 & -9 & 2.5 & 1\\
   0 & 0 & 0 & 0 & 0
\end{bmatrix} \begin{bmatrix}
                1 & 0 & 0 & 0 & 0\\
                0 & 0 & 0 & 1 & 0\\
                0 & 0 & 1 & 0 & 0\\
                0 & 1 & 0 & 0 & 0\\
                0 & 0 & 0 & 0 & 1
              \end{bmatrix} \begin{bmatrix}
                                1 & 0 & 0 & 0 & 0\\
                                0 & 0 & 0 & 1 & 0\\
                                0 & 0 & 1 & 0 & 0\\
                                0 & 1 & 0 & 0 & 0\\
                                0 & 0 & 0 & 0 & 1
                              \end{bmatrix}
)
\]
\end{document}
