\documentclass{article}

\usepackage{fancyhdr}
\cfoot{
\vspace{1mm}\hspace{3cm}
\includegraphics[width=0.2\textwidth]{CC-BY-NC-SA.pdf}}

\renewcommand{\headrulewidth}{0pt}
\renewcommand{\footrulewidth}{0pt}
\setlength\headheight{80.0pt}
\addtolength{\textheight}{-80.0pt}

\usepackage[margin = 3cm, footskip = 30pt]{geometry}
\usepackage{amsmath}
\usepackage{blkarray}
\usepackage[table]{xcolor}
\usepackage{amssymb}
\usepackage{amsfonts}
\usepackage{enumerate}
\newcommand\bg{\cellcolor{gray!70}}

\usepackage{stackengine,graphicx}
\def\stacktype{L}
\def\useanchorwidth{T}
\newcommand\strike[1]{\stackon[3.3pt]{#1}{\rule{4.5ex}{1pt}}}
\newcommand\vstrike[1]{\stackon[0pt]{#1}{\smash{\rule[-3pt]{1pt}{2.9ex}}}}

\usepackage{tikz}
\usetikzlibrary{shapes.geometric, arrows,positioning,automata}


\makeatletter
\renewcommand*\env@matrix[1][*\c@MaxMatrixCols c]{
  \hskip -\arraycolsep
  \let\@ifnextchar\new@ifnextchar
  \array{#1}}
\makeatother

\title{Linear Mapping (Part 008)}
\author{Dr Kapil\\kapil $@$ nitkkr $\cdot$ ac $\cdot$ in\\ Department of Computer Applications\\ NIT Kurukshetra}
\date{\today}

\begin{document}
\maketitle
\thispagestyle{fancy}

Mapping is another word to refer a function. Now, we want to study those mapping which when used on vectors do not change there property of linearity (the core property).

i.e. if \( \Phi : V \rightarrow W \), where V, W are real vector spaces. then the function $\Phi$ preserves the structure of the vector space if
\begin{align}
\Vec{\Phi}(\Vec{x} + \Vec{y}) &= \Vec{\Phi}(\Vec{x}) + \Vec{\Phi}(\Vec{y}) ~~~~~~~~~~~~~~~\forall \Vec{x},\Vec{y} \in V \nonumber \\
\Vec{\Phi}(\lambda \Vec{x}) &= \lambda \Vec{\Phi}(\Vec{x}) ~~~~~~~~~~~~~~~~ \forall \Vec{x} \in V ~~ \& ~~ \forall \lambda \in \mathbb{R} \nonumber
\end{align}

This can be summarized as follows - \\
For a vector spaces V, W a mapping $\Phi$: V $\rightarrow$ W is known as Linear Mapping (or Vector Space homomorphism  $\backslash$ Linear Transformation) if 
\begin{align}
    \forall \Vec{x}, \Vec{y} \in V \forall \lambda , \Phi \in \mathbb{R} \nonumber \\
    \Phi (\lambda \Vec{x} + \psi \Vec{y}) = \lambda \Phi (\Vec{x}) + \psi \Phi (\Vec{y)} \nonumber
\end{align}

We will see that any linear mapping can be written in the form of matrices.
So, first let us check if matrices shows Linear Mapping or not\\

\end{document}